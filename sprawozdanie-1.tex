\documentclass[10pt,a4paper]{article}
\usepackage[a4paper]{geometry}

\usepackage{polski}
\usepackage{xltxtra}
\usepackage{indentfirst}
\usepackage{relsize}
\usepackage{fancyvrb}
\usepackage{hyperref}
\hypersetup{
    pdftitle={Sprawozdanie z ćwiczenia nr 1 z laboratorium Programowanie internetowe},%
    pdfauthor={Tomasz Cudziło},%
    colorlinks=true,        % false: boxed links; true: colored links
    linkcolor=black,        % color of internal links
    citecolor=green,        % color of links to bibliography
    filecolor=magenta,      % color of file links
    urlcolor=cyan,          % color of external links
    unicode=true,           % non-Latin characters in Acrobat’s bookmarks
    pdfstartview={FitH},    % fits the width of the page to the window
    pdfnewwindow=true       % links in new window
}

%% tweak fonts
\defaultfontfeatures{Mapping=tex-text}
\setromanfont{Charis SIL}
\setsansfont[Scale=MatchLowercase]{Helvetica Neue}
\setmonofont[Scale=MatchLowercase]{Menlo}
\linespread{1.25}

%% define custom commands and environments
\DefineVerbatimEnvironment%
  {SmallVerbatim}%
  {Verbatim}{fontsize=\relsize{-0.5},numbers=left,numbersep=-10pt,frame=lines,tabsize=4}

\newcommand{\f}[1]{\texttt{#1}}
\newcommand{\s}[1]{\textsf{#1}}

\begin{document}

%%fakesection{Tytuł}
\title{
  Sprawozdanie z~ćwiczenia nr~1\\z~laboratorium Programowanie internetowe
}
\author{
  Tomasz Cudziło\\
  \textsc{PW EE Informatyka}\\[10pt]
}
\date{\today}
\maketitle

\section{Opis projektu}
Celem ćwiczenia było zapoznanie się z~językami \f{HTML 4.01} oraz \f{XHTML 1.0},
wykorzystanie elementów przez nie oferowanych oraz dostrzeżenie
i~zaprezentowanie różnic między nimi.

\subsection{Wykonanie}
\begin{description}
  \item[Adres projektu:] \hfill \\
  \url{http://volt.iem.pw.edu.pl/~cudzilot/pi/cw1/index.html}
  \item[Repozytorium projektu] \hfill \\
  \url{https://github.com/student-tomasz/pi-cwiczenie-1}
  \item[Repozytorium sprawozdania:] \hfill \\
  \url{https://github.com/student-tomasz/pi-sprawozdanie-1}
\end{description}

\subsection{Narzędzia}
Kod został napisany w~\f{Vim}, bez wykorzystania generatorów\footnote{Oprócz
tagu \f{<embed>} w~liście elementów \f{HTML}, który został skopiowany ze strony
filmu.}. Zarządzanie kodem odbywa się za pomocą \f{git} i~serwisu
\f{\href{https://github.com/}{GitHub}}. Również aktualizacja źródeł na serwerze
wykorzystuje \f{git}.

\subsection{Biblioteki}
Strony posiadają \f{JavaScript} korzystający z~biblioteki
\f{\href{http://jquery.com/}{jQuery}} w~wersji \f{1.6.4}, oraz
\f{\href{http://code.google.com/p/google-code-prettify/}{google-code-prettify}}
w~wersji \f{release}~z~dnia \f{2011-06-01}.

Wykorzystywane też są style \f{CSS} normalizujące wyświetlanie elementów.
Pochodzą one z~biblioteki \f{\href{http://yuilibrary.com/}{YUI 3}}. Są to pliki
\f{reset.css}, \f{base.css} i~\f{fonts.css} w~katalogu
\f{\href{https://github.com/student-tomasz/pi-laboratoria/tree/master/css}{../css/}}
w~stosunku do katalogu zawierającego projekt z~ćwiczenia.

\end{document}

