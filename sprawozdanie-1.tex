\documentclass[10pt,a4paper]{article}
\usepackage[a4paper]{geometry}

\usepackage{polski}
\usepackage{xltxtra}
\usepackage{indentfirst}
\usepackage{relsize}
\usepackage{fancyvrb}
\usepackage{hyperref}
\hypersetup{
    pdftitle={Sprawozdanie z ćwiczenia nr 1 z laboratorium Programowanie internetowe},%
    pdfauthor={Tomasz Cudziło},%
    colorlinks=true,        % false: boxed links; true: colored links
    linkcolor=black,        % color of internal links
    citecolor=green,        % color of links to bibliography
    filecolor=magenta,      % color of file links
    urlcolor=cyan,          % color of external links
    unicode=true,           % non-Latin characters in Acrobat’s bookmarks
    pdfstartview={FitH},    % fits the width of the page to the window
    pdfnewwindow=true       % links in new window
}

%% tweak fonts
\defaultfontfeatures{Mapping=tex-text}
\setromanfont{Charis SIL}
\setsansfont[Scale=MatchLowercase]{Helvetica Neue}
\setmonofont[Scale=MatchLowercase]{Menlo}
\linespread{1.25}

%% define custom commands and environments
\DefineVerbatimEnvironment%
  {SmallVerbatim}%
  {Verbatim}{fontsize=\relsize{-0.5},numbers=left,numbersep=-10pt,frame=lines,tabsize=4}

\newcommand{\f}[1]{\texttt{#1}}
\newcommand{\s}[1]{\textsf{#1}}

\begin{document}

%%fakesection{Tytuł}
\title{
  Sprawozdanie z~ćwiczenia nr~1\\z~laboratorium Programowanie internetowe
}
\author{
  Tomasz Cudziło\\
  \textsc{PW EE Informatyka}\\[10pt]
}
\date{\today}
\maketitle



\section{Opis projektu}
Celem ćwiczenia było zapoznanie się z~językami \f{HTML 4.01} oraz \f{XHTML 1.0},
wykorzystanie elementów przez nie oferowanych oraz dostrzeżenie
i~zaprezentowanie różnic między nimi.

\subsection{Wykonanie}
\begin{description}
  \item[Adres projektu:] \hfill \\
  \url{http://volt.iem.pw.edu.pl/~cudzilot/pi/cw1/index.html}
  \item[Repozytorium projektu] \hfill \\
  \url{https://github.com/student-tomasz/pi-cwiczenie-1}
  \item[Repozytorium sprawozdania:] \hfill \\
  \url{https://github.com/student-tomasz/pi-sprawozdanie-1}
\end{description}

\subsection{Narzędzia}
Kod został napisany w~\f{Vim}, bez wykorzystania generatorów\footnote{Oprócz
tagu \f{<embed>} w~liście elementów \f{HTML}, który został skopiowany ze strony
filmu.}. Zarządzanie kodem odbywa się za pomocą \f{git} i~serwisu
\f{\href{https://github.com/}{GitHub}}. Również aktualizacja źródeł na serwerze
wykorzystuje \f{git}.

\subsection{Biblioteki}
Strony posiadają \f{JavaScript} korzystający z~biblioteki
\f{\href{http://jquery.com/}{jQuery}} w~wersji \f{1.6.4}, oraz
\f{\href{http://code.google.com/p/google-code-prettify/}{google-code-prettify}}
w~wersji \f{release}~z~dnia \f{2011-06-01}.

Wykorzystywane też są style \f{CSS} normalizujące wyświetlanie elementów.
Pochodzą one z~biblioteki \f{\href{http://yuilibrary.com/}{YUI 3}}. Są to pliki
\f{reset.css}, \f{base.css} i~\f{fonts.css} w~katalogu
\f{\href{https://github.com/student-tomasz/pi-laboratoria/tree/master/css}{../css/}}
w~stosunku do katalogu zawierającego projekt z~ćwiczenia.

\subsection{Budowa dokumentów oparta o~elementy \f{<div>} i~\f{<table>}}
Strony miały zachowują się i~wyglądają identycznie, niezależnie od przyjętej
metody pozycjonowanie elementów. Ewentualne różnice wynikają z~różnic
w~implementacji standardów przez przeglądarki i~są opisane w~paragrafie
\ref{sec:uwagi_budowa}. Ogólny projekt zakłada:
\begin{itemize}
  \item nagłówek,
  \item treść:
  \begin{itemize}
    \item z~nawigacją po lewej stronie, zajmującej około $30\%$ szerokości
      rodzica,
    \item i~porównaniem po prawej stronie, zajmującym resztę wolnej szerokości,
  \end{itemize}
  \item stopkę,
  \item źródła.
\end{itemize}



\section{Wnioski}
\subsection{Elementy \f{HTML}}
Język \f{HTML} pozwala na tworzenie rozbudowanych dokumentów, które przy
stosowaniu się do standardów \f{W3C} oferują identyczne doświadczenie
użytkownikowi niezależnie od jego urządzenia i~systemu operacyjnego.

Ponieważ dostęp do Internetu jest powszechny i~nowoczesne przeglądarki wspierają
standardy \f{W3C}, jest to wygodna platforma do tworzenia aplikacji.

\subsection{Porównanie języków}
Język \f{XHTML} powstał by znormalizować \f{HTML} na tyle by parsery \f{XML}
mogły swobodnie przetwarzać dokumenty. Nie rozszerza jego funkcjonalności
w~znaczy sposób, wymusza precyzyjną składnię i~rozdziela warstwę prezentacji
od danych~--- tj. usuwa atrybuty i~tagi z~\f{HTML}, które odpowiadają za wygląd
a~nie zawierają treści.

\subsection{Budowa dokumentów oparta o~elementy \f{<div>} i~\f{<table>}}
Tworzenie strony, której layout jest stworzony na bazie \f{<table>} wymaga
więcej czasu i~kodu niż stworzenie identycznego wyglądu na bazie \f{<div>}.
Strony porównań są identyczne względem ich wyglądu. Strony z~tabelami zawierają
około $40\%$ linii niż ich odpowiedniki z~\f{<div>}.

Ustawianie elementów strony za pomocą tabel jest zupełnie nie semantyczne,
bezpośrednio wymusza ich pozycję w~dokumencie \f{HTML}. Funkcjonalność ta jest
spełniana stylami \f{CSS}. W~efekcie nadmiarowa ilość znaczników zamykających
komórki i~wiersze tabel dodatkowo zmniejsza czytelność kodu i~powoduje raka
płuc. \emph{Tabele służą tylko do prezentowania zestawu danych.}

Technika tworzenia stron w~oparciu o~tabele była popularna kiedy obowiązujące
standardy \f{HTML} i~\f{CSS} nie dawały innych możliwości. To było w~1996 roku.

Layout na \f{<div>} jest łatwiejszy w~tworzeniu i~utrzymaniu. Zmiany przeważnie
wprowadza się tylko w~stylach, kod dokumentu jest przejrzysty i~semantycznie
poprawny.



\section{Uwagi}
Wszystkie strony wymagają włączonej obsługi \f{JavaScript}. Wszystkie są zgodne
ze standardami \f{HTML~4.01~Strict} lub \f{XHTML~1.0~Strict}, ich style
z~\f{CSS~2.1}.

Strony zostały przetestowane w~\f{Firefox 7.0.1}, \f{Safari 5.1.1} i \f{Chrome
15.0}.

\subsection{Elementy \f{HTML}}
Strona zawierająca demonstracyjne wykorzystanie elementów \f{HTML} jest
niezgodna ze standardem ze swojej deklaracji \f{DOCTYPE}. Zawiera elementy
przedstawione na wykładzie, nie wchodzące w~skład specyfikacji \f{XHTML 1.0
Strict}.

\subsection{Budowa dokumentów oparta o~elementy \f{<div>} i~\f{<table>}}
\label{sec:uwagi_budowa}
Strony wyglądają identycznie. W~\f{Operze 11.52} margines dolny tabeli jest
traktowany niż w~silnikach \f{Gecko} lub \f{Webkit} i~jest doliczany do odstępu
marginesu górnego elementu \f{<h2>} walidacji. Stąd drobna różnica w~odstępie.

\subsection{Wyświetlanie źródła strony}
Wyświetlanie źródła strony jest oparte na \f{JavaScript}. Kod wstawiający źródło
na stronę jest umieszczony w~pliku
\f{\href{https://github.com/student-tomasz/pi-laboratoria/blob/master/js/insert-source.js}{../js/insert-source.js}}
w~stosunku do indexu projektu.

Gdy strona zostanie wczytana i~\f{DOM} załadowany, wysyłane są żądania po pliki
źródłowe wykorzystywane na aktualnie przeglądanej stronie. Najpierw zbierane są
ścieżki do plików podlinkowanych na stronie, oraz ścieżka samej strony.
Następnie wysyłane jest żądanie po każdy z~tych plików, którego odpowiedź
interpretowana jest jako zwykły tekst.

Żądania są wysyłane synchronicznie, może się zdarzyć, że przedłuży to znacznie
czas ładowania strony. Nowoczesne przeglądarki zachowują załadowane zasoby
w~pamięci podręcznej, z~której powinny być wczytywane odpowiedzi na następne
zapytania.

\end{document}
